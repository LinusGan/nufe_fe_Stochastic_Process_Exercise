\documentclass{article}
\usepackage[UTF8]{ctex}
\usepackage{graphicx} % Required for inserting images
\usepackage{amsmath}
\usepackage{amssymb}
\usepackage{geometry}
\geometry{a4paper, margin=1in}


\begin{document}
\title{Stochastic-Process-Exercise}
\author{dashun gan}
\date{October 2024}
\maketitle
\newpage


\section*{第二章}
\subsection*{1}
重复地抛掷一枚均匀的硬币,抛掷结果为 $Y_0, Y_1, Y_2, \dots$,它们取值为 $0$ 或 $1$ 的概率均为 $1/2$,用 $X_n = Y_n + Y_{n-1}$ ($n \geq 1$)表示第 $(n-1)$ 次和第 $n$ 次抛掷出的结果中 $1$ 的个数。$X_n$ 是一个马尔可夫链吗?

\vspace{3cm}
\subsection*{2}
小王家每天早晨都会收到报纸并且看完之后将它们堆放起来。每天傍晚,有人把所有堆起来的报纸拿走放到回收箱的概率为 $1/3$。另外,如果堆起来的报纸至少有 $5$ 张的话,小王将以概率 $1$ 把报纸放到回收箱中,考虑晚上堆起来的报纸数。

求相应的状态空间和转移概率矩阵。
\vspace{3cm}
\subsection*{3}
五个白球和五个黑球分散在两个罐子中,其中每个罐子中都有五个球。每一次我们各从两个罐子中随机抽取一个球并交换它们。用 $X_n$ 表示在时刻 $n$ 左边罐子中白球的个数。

求 $X_n$ 的转移概率及对应的转移概率矩阵。
\vspace{6cm}
\subsection*{5}
五个白球和五个黑球分散在两个罐子中,其中每个罐子中都有五个球。每一次我们各从两个罐子中随机抽取一个球并交换它们。用 $X_n$ 表示在时刻 $n$ 左边罐子中的白球的个数。

求 $X_n$ 的转移概率及对应的转移概率矩阵。

\vspace{3cm}
\subsection*{8}
一个出租车司机在机场 $A$ 和宾馆 $B$、宾馆 $C$ 之间按照如下方式行车:如果他在机场,那么下一时刻他将以等概率到达两个宾馆中的任意一个;如果他在其中一个宾馆,那么下一时刻他以概率 $3/4$ 返回机场,以概率 $1/4$ 开往另一个宾馆。

假设时刻 $0$ 司机在机场,分别求出时刻 $2$ 司机在这三个可能地点的概率以及时刻 $3$ 他在宾馆 $B$ 的概率。
\vspace{4cm}
\subsection*{12}
给出下列马尔可夫链的平稳分布,其转移矩阵为:
\[
\text{(a)} \quad 
\begin{bmatrix}
0.5 & 0.4 & 0.1 \\
0.2 & 0.5 & 0.3 \\
0.1 & 0.3 & 0.6
\end{bmatrix}, \quad
\text{(b)} \quad 
\begin{bmatrix}
0.5 & 0.4 & 0.1 \\
0.3 & 0.4 & 0.3 \\
0.2 & 0.2 & 0.6
\end{bmatrix}, \quad
\text{(c)} \quad 
\begin{bmatrix}
0.6 & 0.4 & 0 \\
0.2 & 0.4 & 0.4 \\
0 & 0.2 & 0.8
\end{bmatrix}.
\]

\vspace{6cm}
\subsection*{13}
考虑状态空间为 $S = \{0, 1, \dots, 5\}$ 上的一个马尔可夫链,其转移概率矩阵为:
\[
\begin{bmatrix}
0.5 & 0.5 & 0   & 0   & 0   & 0   \\
0.3 & 0.7 & 0   & 0   & 0   & 0   \\
0   & 0   & 0.1 & 0   & 0.9 & 0   \\
0.25& 0.25& 0   & 0   & 0.25 & 0.25 \\
0   & 0   & 0.7 & 0   & 0.3 & 0   \\
0   & 0.2 & 0   & 0.2 & 0.2 & 0.4
\end{bmatrix}
\]
\begin{enumerate}
    \item 互通类有哪些?
    \item 常返态有哪些?
    \item 非常返态又有哪些?
\end{enumerate}

\vspace{5cm}
\subsection*{15}
一个大学提供三种类型的健康计划:A、B 和 C。经验显示,人们依照下面的转移概率矩阵改变健康计划:
\[
\begin{bmatrix}
0.85 & 0.1 & 0.05 \\
0.2 & 0.7 & 0.1 \\
0.1 & 0.3 & 0.6
\end{bmatrix}.
\]
在 2020 年,选择这三种计划的人员占的比例分别是 $30\%$、$25\%$ 和 $45\%$。
\begin{enumerate}
    \item 2021 年选择这三种计划的人员占的比例分别是多少?
    \item 从长远来看,选择这三种计划的人员占的比例分别是多少?
\end{enumerate}
\pagebreak
\section*{第四章习题}
\subsection*{1}
假设维修一台机器的时间可用一个服从均值为 2 的指数分布的随机变量来描述,请问:
\begin{enumerate}
    \item 维修机器花费的时间在 2 小时以上的概率是多少?
    \item 在已知维修机器要花费 3 小时以上的条件下,花费的时间超过 5 小时的概率是多少?
\end{enumerate}
\vspace{3cm}
\subsection*{2}
一台收音机的寿命服从均值为 5 年的指数分布,如果购买一部已经使用了 7 年的收音机,那么它还能继续工作 3 年的概率是多少?
\vspace{3cm}
\subsection*{4}
A 和 B 同时进入一家美容院,A 要修指甲,而 B 要理发。假定修指甲(理发)的时间服从均值为 20(30)分钟的指数分布,请问:
\begin{enumerate}
    \item A 先修完指甲的概率是多少?
    \item 直到 A 和 B 都完成要花费的时间的期望是多少?
\end{enumerate}
\vspace{7cm}
\subsection*{15}
假定某品牌的灯泡次品率是 $1\%$。在装有 $25$ 个灯泡的产品中,运用泊松分布来近似计算最多有一个次品的概率。
\vspace{3cm}
\subsection*{16}
假定 $N(t)$ 是速率为 $3$ 的泊松过程,令 $T_n$ 表示第 $n$ 个到达的时刻,求:
\begin{enumerate}
    \item $\mathbb{E}[T_{12}]$;
    \item $\mathbb{E}[T_{12} \mid N(2) = 5]$;
    \item $\mathbb{E}[N(5) \mid N(2) = 5]$。
\end{enumerate}
\vspace{4cm}
\subsection*{18}
假设某接听呼叫的服务台每小时接到的呼叫数是一个速率为 $4$ 的泊松过程。
\begin{enumerate}
    \item 在第一个小时内呼叫数少于 $2$ 个的概率是多少?
    \item 假定在第一个小时内有 $6$ 个呼叫,求在第三个小时呼叫数少于 $2$ 个的概率。
    \item 假定服务台的话务员接听 $10$ 个呼叫后需要休息一下,那么她的平均工作时间是多少?
\end{enumerate}
\vspace{7cm}
\subsection*{23}
事件按速率为每小时 $\lambda = 2$ 的泊松过程发生,请问:
\begin{enumerate}
    \item 在晚上 8 点到 9 点没有事件发生的概率是多少?
    \item 从正午开始,到第 4 个事件发生的期望时间是多少?
    \item 在晚上 6 点到 8 点有两个或两个以上事件发生的概率是多少?
\end{enumerate}
\vspace{6cm}
\subsection*{26}
一家保险公司的赔付数是一个速率为每周 $4$ 单的泊松过程。将“千元”简记为 $K$,假设每个保险单的赔付金额均值为 $10K$,标准差为 $6K$。求 $4$ 周赔付的总金额的均值和标准差。
\pagebreak
\pagebreak
\section*{第五章习题}
\subsection*{1}
一小间办公室中有两个人进行股票共同基金的销售业务,他们每个人的状态有两
种:要么在打电话,要么没在打电话。假设业务员$i$的通话时间服从速率为$\mu_i$的
指数分布,没在打电话的时间服从速率为$\lambda_i$的指数分布。\\
构建一个马氏链模型,状态空间为$\{0,1,2,12\}$,其中状态表示正在打电话的业务员。
\vspace{6cm}
\subsection*{3}
考虑以下情况:有两台机器,仅有一位维修工人负责维修。机器$i$在发生故障前可正常工作的时间服从速率为$\lambda_i$的指数分布,每台机器的维修时间服从速率为$\mu_i$的指数分布,且维修工人依照机器发生故障的次序进行维修。\\
$(a)$构建一个此情形下的马氏链,其状态空间为$\{0,1,2,12,21\}$;\\
$(b)$假定$\lambda_1=1,\mu_1=2,\lambda_2=3,\mu_2=4,$求平稳分布。
\vspace{8cm}
\subsection*{6}
一小间办公室中有两个人进行股票共同基金的销售业务,他们每个人的状态有两
种:要么在打电话,要么没在打电话。假设业务员的通话时间均服从速率为$3$的指数分布,没在打电话的时间均服从速率为$1$的指数分布。\\
$(a)$求平稳概率分布。\\
$(b)$假定他们升级了电话系统,若打入的电话正在通话中,则转接到另一个电话,但若另一个电话也正在通话中,则打不进去电话,求新的平稳概率分布。
\vspace{7cm}
\subsection*{8}
一台机器容易发生$i=1,2,3$三种故障,发生速率分别为$\lambda_i$,维修它们需要在费的时间服从速率为$\mu_i$的指数分布。\\
构建一个状态空间为$\{0,1,2,3\}$的马氏链并求其平稳分布。
\vspace{9cm}
\subsection*{9}
三只青蛙在池塘附近玩耍。当它们在地面上晒太阳时,它们觉得太热了,于是以速率1眺入池塘:当它们在池塘中时,它们觉得太冷了,于是以速率2跳回地面上。
用$X_t$表示时刻$t$晒太阳的青蛙数。\\
求$X_t$的平稳分布。
\vspace{5cm}

\pagebreak
\section*{第六章习题}
\subsection*{1}
已知$W_t$是标准布朗运动,假设$X(t)=|W(t)|,t \ge 0$,求$\mathbb{E}X_t$和$VarX(t)$。
\vspace{4cm}
\subsection*{2}
假设$W_t$是标准布朗运动,求:\\
$(a)$ $\mathbb{P}[W(2)>3];$\\
$(b)$ $\mathbb{P}[W(3)>W(2)]。$
\vspace{3cm}
\subsection*{3}
假设$W_t$是标准布朗运动,求:\\
$(a)$ $aW(s)+bW(t)$的分布,其中$a,b,s,t$均是实数,并且$0<s<t$;\\
$(b)$ $\mathbb{P}[W(2)-2W(3) \le 4]。$
\vspace{7cm}
\subsection*{4}
假设$W_t$是标准布朗运动,并且$0 \le u \le s \le t$,求:\\
$(a)$ $\mathbb{E}[W^2(t)W^2(s)]$\\
$(b)$ $\mathbb{E}[W(t)W(s)W(u)]。$
\vspace{7cm}
\subsection*{5}
假设$W_t$是标准布朗运动,求$W(1)+W(2)+ \dots +W(n)$的分布。
\pagebreak
\section*{第六章}
\subsection*{1}
已知 $W(t)$ 是标准布朗运动,假设 $X(t) = |W(t)|, \; t \geq 0$,求 $\mathbb{E}[X(t)]$ 和 $\operatorname{Var}[X(t)]$。
\vspace{5cm}
\subsection*{2}
假设 $W(t)$ 是标准布朗运动,求:
\begin{enumerate}
    \item $\mathbb{P}[W(2) > 3]$;
    \item $\mathbb{P}[W(3) > W(2)]$。
\end{enumerate}
\vspace{2cm}
\subsection*{3}
假设 $W(t)$ 是标准布朗运动,求:
\begin{enumerate}
    \item $aW(s) + bW(t)$ 的分布,其中 $a, b, s, t$ 均是实数,并且 $0 < s < t$;
    \item $\mathbb{P}[W(2) - 2W(3) \leq 4]$。
\end{enumerate}
\vspace{7cm}
\subsection*{4}
假设 $W(t)$ 是标准布朗运动,并且 $0 \leq u \leq s \leq t$,求:
\begin{enumerate}
    \item $\mathbb{E}[W^2(t)W^2(s)]$;
    \item $\mathbb{E}[W(t)W(s)W(u)]$。
\end{enumerate}
\vspace{6cm}
\subsection*{5}
假设 $W(t)$ 是标准布朗运动,求 $W(1) + W(2) + \cdots + W(n)$ 的分布。
\pagebreak
\section*{第七章}
\subsection*{1}
令 $S_n = X_1 + X_2 + \cdots + X_n$,其中 $X_i$ 相互独立,并且 $\mathbb{E}[X_i] = 0, \; \operatorname{Var}(X_i) = \sigma^2$。

证明:$S_n^2 - n\sigma^2$ 是一个鞅。
\vspace{6cm}
\subsection*{5}
假设 $W(t)$ 是标准布朗运动,$\lambda$ 和 $v$ 均是实数,且 $v \geq 0$,令:
\[
X(t) = \exp[\lambda W(t) - vt]
\]
证明:当 $\lambda^2 = 2v$ 时,$X(t)$ 是一个鞅。
\vspace{10cm}
\subsection*{6}
假设 $W(t)$ 是标准布朗运动,$X(t) = W(t) + \mu t$,令:
\[
M(t) = \exp[-2\mu X(t)]
\]
证明:$M(t)$ 是一个鞅。
\vspace{7cm}
\subsection*{7}
假设 $W(t)$ 是标准布朗运动,令:
\[
M(t) = W^3(t) - 3tW(t)
\]
证明:$M(t)$ 是一个鞅。
\vspace{6cm}
\pagebreak
\section*{第八章}
\subsection*{10}
已知 $X(t)$ 对应的随机微分方程如下:
\[
\mathrm{d}X(t) = -\frac{1}{2}\theta^2(t) \, \mathrm{d}t - \theta(t) \, \mathrm{d}W(t)
\]
其中,$W(t)$ 是标准布朗运动。

求 $Z(t) = \exp[X(t)]$ 的随机微分方程。
\vspace{8cm}
\subsection*{11}
已知 $X(t)$ 是一个 O-U 过程,其对应的随机微分方程如下:
\[
\mathrm{d}X(t) = -\kappa X(t) \, \mathrm{d}t + \sigma \, \mathrm{d}W(t)
\]
其中,$W(t)$ 是标准布朗运动。

假设 $Y(t) = X^2(t)$,求 $Y(t)$ 的随机微分方程。
\vspace{8cm}
\subsection*{12}
若
\[
\mathrm{d}X(t) = \mu X(t) \, \mathrm{d}t + \sigma X(t) \, \mathrm{d}W(t),
\]
求:
\begin{enumerate}
    \item $X^k(t)$ 的随机微分方程;
    \item $X^{-1}(t)$ 的随机微分方程。
\end{enumerate}
\vspace{8cm}
\subsection*{15}
已知 $X(t)$ 服从几何均值回复过程(geometric mean-reverting process),其随机微分方程如下:
\[
\mathrm{d}X(t) = \kappa [\theta - \ln X(t)] X(t) \, \mathrm{d}t + \sigma X(t) \, \mathrm{d}W(t)
\]
求 $Y(t) = \ln X(t)$ 的随机微分方程。
\pagebreak
\end{document}
